%the body as a home for the picaro. Must a picaro be in tune with his body?
\documentclass{article}
\title{Keyword project Entry 5}
\author{Mackenzie Norman}
\begin{document}
\maketitle

In reading Animals People and our class discussion about the genre of picaresque (the first time I had heard of such a thing) I was thinking about the role of body in terms of the picaresque - especially the body of the picaro. I mostly approach this from the perspective of Animal from the aforementioned animals people and Bukowski's alter-ego Henry Chianski in Factotum -- or really any work by Bukowski, he quite enjoys painting himself as a picaro -- mostly because both characters are fond of bragging about their impressive ``Lund''. 

For both Chianski and Animal, their body, inspite of its unpleasant appearance represents a home of sorts. As picaros, their living situations are often non-existent or very meager, so they are forced to find comfort in their body being their only consistent `home'. Despite this, both characters mistreat their bodies as well, Chianski with booze and Animal with reckless behaviour such as climbing a sharp mango tree. This rejection of care for their body helps endear the picaro to the reader, since so many of us have various neurosis around our body and its inevitable deterioration. So to see a character so free of this fear - in spite of the obvious damage being done is quite endearing. 
\end{document}