\documentclass{article}
\usepackage{hyperref}
\title{Keyword project entry 1}
\author{Mackenzie Norman}
\date{}
\begin{document}
\maketitle
``The question what is a body, is the question what is it to eat: Take, eat; this is my body.'' Brown \\


When I think of \textit{body} in terms of colonialism I first think of the human body, and the legislation and control often exerted on bodies by colonial powers. I tend to think of Quentin Compsons obession with the purity of his sisters body in \textit{The Sound and The Fury} which you can read it as colonialism (is the old confederate south not a bastion of colonialism) being one of the multitude of anarochisms that tear Quentin apart. I chose body because I think it is one of the most interesting things that is legislated , etc. How can something that is only yours , only touches you, only affects you be justifiably controlled, and what ``justification'' have colonial powers used. 
On the flip side of this, I am equally interested in looking at how boodies are used as tools in resistance. The Fall illustrated this well with the use of body as a shield, and in the documentary with the women clashing with the police with their bodies uncovered.


From OED we can see it etymologically germanic, with roots most likely from the old german \textit{botah}. Also mentioned is a connection to the latin \textit{Corpus} especially the idea of a body as a collection i.e \textit{corpus christi} (the collection of writing on christ)

Various Definitions from OED\\

``The physical or mortal nature, state, or aspect of man. Frequently in in (the) body, out of (the) body and variants, sometimes contrasted with in spirit.''\\

``The complete physical form of a person or animal; the assemblage of parts, organs, and tissues that constitutes the whole material organism.''\\

``A united or organized whole; an aggregate of individuals characterized by some common attribute; a collective mass.''\\

``Substance, as opposed to representation, shadow, etc.; reality.''\\

Looking at these definitions, all of them will be very interesting to interrogate as this term continues. Already I feel like I could look at the fall with the definition of collective mass, I am also very interested in contrasting the body with the spirit and will be looking for that in our readings.\\

\section{Citations}

Brown, Norman O.. Love's Body, Reissue of 1966 Edition, University of California Press, 1990. ProQuest Ebook Central, \url{http://ebookcentral.proquest.com/lib/psu/detail.action?docID=837241}.
Created from psu on 2025-04-09 17:51:58. 
\end{document}