\documentclass{article}
\usepackage{hyperref}
\usepackage{setspace}

\title{Keyword project Entry 2}
\author{Mackenzie Norman}
\date{}
\begin{document}
\doublespacing 
\maketitle
\textit{``The question what is a body, is the question what is it to eat: Take, eat; this is my body.''}(Brown) \\


For this journal entry I wanted to look at the contrast between the other women and Lucia in \textit{Nervous Conditions}. In the story Lucia was by far the most interesting character to me. She is a foil to every other woman. To the point of nearing masculinity, but despite her masculine qualities, her body is most often described of that of a woman. \textit{``But unlike my mother her complexion always had a light shining from underneath it.'' } I think this is what makes her a novel contradiction. She is feminine in body surely - even becoming pregnant. (However when describing her giving birth it is minimized) ``Lucia had a kid quote''. She encompasses mind-body dualism in an interesting way simply because her body is so divorced from her mind (and her mind is so divorced from reality)

\pagebreak

``You've got a waist. One of these days you'll have a bust. Pity about the backside, ... It's rather large  ''[91]
``Shes probably dieting because I told her her bottom is fat''[92]

``nyasha fell on to the bed, her miniscule skirt riding up to her bottom''[114]
``Lucia had somehow managed to keep herself plump despite her tribulations''[127]
``As if it is ever easy, And these days it is worse, with the poverty of blackness on one side and the weight of womanhood on the other [16]''

Colonialism puts womens bodies in a constant state of contradiction  - plump but not fat... etc.
\paragraph{}

\section{Citations}

Brown, Norman O.. Love's Body, Reissue of 1966 Edition, University of California Press, 1990. ProQuest Ebook Central, \url{http://ebookcentral.proquest.com/lib/psu/detail.action?docID=837241}.
Created from psu on 2025-04-09 17:51:58. 
\end{document}