\documentclass{article}
\title{Entry 7}
\author{Mackenzie Norman}
\begin{document}
\maketitle
\textit{``From crucifixion to eucharist; from the blood to the bloodless sacrifice; from sacrifice to feeding. The solution to the problem of war. Bread and wine, this is my body''}
\paragraph{}
 Natalie Diaz begins by writing ``I’ve been taught bloodstones can cure a snakebite,
can stop the bleeding'', immediately situating the body as nature, and creating an interesting idea that a stone of blood can stop bleeding. Through out the whole poem she situates the body as nature and vice-versa , showing that the body is of the same natural things ``Your hips are quartz ... two rose-horned rams ascending a soft desert wash '', both the internal and external of the body are natural. This poem is intensely physical, any mention of the mind is placed squarely in the memory of a physical process: ``I learned  \textit{Drink} in a country of drought. '' 

Continuing, she says:

``The seeds sleep like geodes beneath hot feldspar sand

until a flash flood bolts the arroyo, lifting them

in its copper current, opens them with memory–

they remember what their god whispered

into their ribs: Wake up and ache for your life. ''

The ribs are some of the most important parts of the human body, especially when you separate `the head from the heart' the ribs protect all that makes a body alive -- mechanically speaking. As the idea of the body being a mechanism, the usage of whispered implies breath; a whisper is on the breath - and so god breathes into their ribs -- their lungs. A myth of creation in essence. 
\end{document}