

\maketitle
\textit{``The divorce between soul and body takes the life out of the body, reducing the organism to a mechanism, dead in itself but given an artificial life, an imitation of life, by will or power: sovereignty is an artificial soul, giving life and motion to the whole body. ''} (O' Brown)
\paragraph{}

For this journal entry I wanted to look at the contrast between the other women and Lucia in \textit{Nervous Conditions}. In the story Lucia was by far the most interesting character to me. She is a foil to every other woman. To the point of nearing masculinity, but despite her masculine qualities, her body is most often described of that of a woman. \textit{``But unlike my mother her complexion always had a light shining from underneath it.'' } I think this is what makes her a novel contradiction. She is feminine in body surely - even becoming pregnant. ``Lucia had needs as all women do'' However when describing her giving birth it is minimized ``Lucia, who by this time had acquired the status of mother'' .  She encompasses mind-body dualism in an interesting way simply because her body is so divorced from her mind (and her mind is so divorced from the lived reality of patriarchy). This is fascinating because it results in her near infiltrating the staunch patriarchy of Babamakuru in a way that none of the other women in the novel can. She especially contrasts with Nyasha, who despite being better educated uses her body as a weapon, whereas I argue that Lucia, eschew the typical divide and contradiction inherent to dualism, is able to be far more successful by being whole.





