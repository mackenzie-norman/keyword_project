%the body as a home for the picaro. Must a picaro be in tune with his body?

``\textit{The world annihilated, the destruction of illusion. The world is the veil we spin to hide the void. The destruction of what never existed. The day breaks, and the shadows flee away}'' (O' Brown)

In reading Animals People and our class discussion about the genre of picaresque (the first time I had heard of such a thing) I was thinking about the role of body in terms of the picaresque - especially the body of the picaro. I mostly approach this from the perspective of Animal from the aforementioned animals people and Bukowski's alter-ego Henry Chianski in Factotum -- or really any work by Bukowski, he quite enjoys painting himself as a picaro -- mostly because both characters are fond of bragging about their impressive ``Lund''. 

For both Chianski and Animal, their body, inspite of its unpleasant appearance represents a home of sorts. As picaros, their living situations are often non-existent or very meager, so they are forced to find comfort in their body being their only consistent `home'. Despite this, both characters mistreat their bodies as well, Chianski with booze and Animal with reckless behaviour such as climbing a sharp mango tree. This rejection of care for their body helps endear the picaro to the reader, since so many of us have various neurosis around our body and its inevitable deterioration. So to see a character so free of this fear - in spite of the obvious damage being done is quite endearing. 

Looking back to cartesian dualism, these both are interesting characters since they quite embody this (yet also do not). Both are characterized as being incredibly smart, yet neither can remotely control their impulses. However as with Animal on page 244, they humanize themselves by unbelievably controlling themselves, however this often at the cost of what they have led the reader to believe as being their desire. I think some would describe this as being an unreliable narrator, but I think this can in part be attributed not to dishonesty but to the picaros inability to truly introspect on their body. Just as you might not introspect on the needs of your house, they aren't lying; the mind (from where the story is narrated) and the body are simply not in sync. 

