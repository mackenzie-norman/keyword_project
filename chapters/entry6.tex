
\textit{``The woman gave me and I did eat. Eating is the form of sex. Copulation is oral copulation; when the Aranda ask each other, "Have you eaten?" they mean, "Have you had intercourse?" The schizophrenic girl refused to eat; the case of Simone Weil. Eating is the form of war. Human blood is the life and de lightful food of the warrior. Eating is the form of redemption.''}
\paragraph{}

I was shocked (as I think everyone was) by the sexual assault at the end of The Life and Times of Micheal K. Looking at it from a body perspective, There is a question of whether or not K ejaculates at all; ``When it was over'' does not imply nor reject ejaculation, viewing the people as clearly always trying to extract things from K. it seems logical to assume he did.  I think K. being forced to ejaculate is important, From O. Brown ``The blood or the soul is the seminal fluid. The fission, or self-alienation which produces this abstract substance, separate from the body but the life of it, is ejaculation. Ejaculation is fission'' We can read his ejaculation as fission, the soul leaving the body, death even. The question this raises (returning to ejaculation) is was there a `soul' to leave the body of K? Looking at the binary of mind and body that cartesian dualism prescribes K's ability to live off next to no food is much more inline with him being someone who is much more of mind than body, since food is often thought as just fuel for the body. However to contrast this, being imprisoned is quite distressing to K, which implies a deep connection to his body and clashes with the stoic refusal to eat.



